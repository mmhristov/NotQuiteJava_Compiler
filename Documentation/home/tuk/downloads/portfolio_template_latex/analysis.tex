\section{Analysis}
\label{sec:analysis}
As interfaces and classes are quite intertwined in this phase, 
this section we will separate into subsections, each of which covering
the changes to a whole file (or class).

\subsection{ClassType.java}
\label{subsec:classType}
This class is our internal representation of classes and is a subtype of the class \textit{Type}.
It has the following fields:
\begin{itemize}
    \item \textit{String} \textbf{name}: the name of the class.
    \item \textit{ClassType} \textbf{superClass}: the direct super-class of the class.
    \item \textit{List<InterfaceType>} \textbf{implementsInterface}: the interfaces it implements.
    \item \textit{Map<String, NQJVarDecl>} \textbf{fields}: the fields of the class.
    \item \textit{Map<String, NQJFunctionDecl>} \textbf{methods}: the methods of the class.
\end{itemize}
These fields are quite self-explanatory, having in mind that identifiers 
(e.g.\@ class names, method names) are unique in our compiler.

Apart from some "getter" and "setter" methods, there is one noteworthy method, which
overrides its super-class' one, namely \textit{isSubTypeOf(ClassType \textbf{other})}.
This method is important in the analysis phase whenever we need
to check two objects' subtype relation (e.g.\@ assignments).
There are generally two cases of interest, depending on the argument's type
\textbf{other}. Its type is either an interface or another class.
In the case it is an interface, we check whether interface \textbf{other} is contained
in the list of interfaces our current class (\textbf{this} object) implements.
If the argument's type is a class, then we recursively check in the ancestor tree
of the current class whether the argument is present or not.
This can be observed in the code snippet below.
\lstinputlisting[language=java, linerange={87-101}, label={code:classTypeInheritance}]{"../../../../../src/main/java/analysis/ClassType.java"}

\subsection{InterfaceType.java}
Similar to the class \textit{ClassType} presented in \Cref{subsec:classType},
\textit{InterfaceType} is our internal representation of interfaces.
The class is a subtype of \textit{Type} and contains the following fields:
\begin{itemize}
    \item \textit{String} \textbf{name}: the name of the class.
    \item \textit{ClassType} \textbf{classImplementation}: the interface object's current class implementation.
    \item \textit{List<InterfaceType>} \textbf{implementsInterface}: the interfaces it implements.
    \item \textit{Map<String, NQJFunctionDecl>} \textbf{functions}: the methods of the interface.
\end{itemize}
Notably, as we will see later when we observe the changes made to the analysis of assignments,
we save the interface object's class implementation in the field \textbf{classImplementation}.
This field is later used in the translation phase to create the corresponding class object.

The  \textit{isSubTypeOf(ClassType \textbf{other})} function here is quite simple.
As we do not consider sub- or super-interfaces in our compiler, the only comparison to be done is
to check if \textbf{other} is the same interface as the current one (meaning both have the same name).

\subsection{NameTable.java}
After we have covered the internal representation of interfaces and classes, we can proceed to
one of the central files of the analysis phase, namely NameTable.\@java,
in which most of the type-checking and analysis operations of classes and interfaces
are being done.

Let us start with the necessary data structures, which we have defined for our needs, visible in the code snippet below.
The comments in the code explain each of the data structure's use-case.
\lstinputlisting[language=java, linerange={16-26, 29-39}, label={code:nameTableMaps}]{"../../../../../src/main/java/analysis/NameTable.java"}

Now we will look at how these data structures are being filled.
The entry point is the constructor of the \textit{NameTable} object.
The idea was to first fill the maps, storing the \textit{NQJ} classes, named 
\textbf{globalClasses} and \textbf{globalInterfaces} and then use these to
create their internal representations (\textit{ClassType} and \textit{InterfaceType})
and store them into the maps \textbf{classTypes} and \textbf{interfaceTypes}.
One of the reasons why this was done is that it enables us to check whether class identifiers and interface identifiers are unique
when filling the \textit{NQJ} maps.
Of course, we here also do not allow a class to have the same name as an interface and vice versa.

The functions \textit{createAndAddClassType()} and 
\textit{createAndAddInterfaceType()} are responsible for the creation of the internal types.
Let us take a closer look at both.
In \textit{createAndAddInterfaceType()} we create an \textit{InterfaceType} object from
an \textit{NQJInterfaceDecl} object and add it to the \textit{interfaceTypes} map.
Additionally, we check if function names are unique in the current interface declaration
and throw errors accordingly.
The functions \textit{createAndAddClassType()} is a bit more sophisticated, but
follows the same idea as in the interfaces' case. 
We create a \textit{ClassType} object, throwing errors when we encounter
some problems in the class declaration. The class' methods are not
being verified here as we get this more or less "for free" from
the already existing procedure in Analysis.java, which will be covered later.
Errors are thrown here in the following cases:
\begin{itemize}
    \item Field names are not unique.
    \item Method names are not unique
    \item An interface that the class implements is not defined.
    \item The class does not implement all functions defined its interfaces.
    \item The class' super-class is not defined.
    \item An inheritance cycle was detected.
\end{itemize}
Since if the super-class of the current class has not yet been created at the point
of construction, we recursively create the super-class first 
(along with its whole ancestor tree).
By passing a list of classes that represents the inheritance tree, we can
check for inheritance cycles quite easily. 
As this is not part of the examination criteria and was done just
for fun, we will skip further explanations here.

\subsection{Analysis.java}
This is the central entry point of the analysis phase. The class creates the
\textit{NameTable} object and using the visitor paradigm starts "visiting"
each component of the program that is to be analyzed.
The points of interest for us are the visitor methods for the classes
\textit{NQJInterfaceDecl} and \textit{NQJClassDecl}.
Let us now examine these three visitor methods.

The visitor method for \textit{NQJInterfaceDecl} objects' job is to
verify the method declarations that an interface contains.
This is done with the already available in the template method that verifies global functions.
This is possible as interface methods do not contain an implementation (i.e.\@ their bodies are empty),
skipping most of the existing computations and type checking.

In the case of \textit{NQJClassDecl}, its visitor method's idea is similar to the 
visitor for interfaces, with the exception that the class' fields are added
to the context data structure before verifying its methods.

As at the point of parsing we can not distinguish between an identifier that
is a class and an identifier that is an interface, we additionally created
a generated class called \textit{TypeInterfaceOrClass}, which expresses
this intuition. 
This class is then used in the function \textit{Analysis.type(NQJType \textbf{type})}
to retrieve either of those types.

\subsection{ExprChecker.java}
This class is used to type-check all expressions by using the matcher paradigm. 
The necessary adjustments to this file were in the matcher methods for
method calls, class field accesses, new object creations, and new array creations.
An underlying commonality between these methods is the fallback behaviour, which
returns a \textbf{Type.ANY} whenever an error is encountered in order to continue 
finding as much errors as possible.
Let us inspect each of these methods closer.

The matcher method for method calls can be divided into two
parts: receiver checking and method checking. First, the receiver of the called method is being verified. This means
that the type-checker searches for the interface or the class declaration of the receiver.
If it does not succeed, an error is thrown.
If it succeeds in finding the interface or class, the type-checker
searches for a method declaration in the object that corresponds to the given method name and throws an error
if necessary. Once it succeeds, the type-checker continues with the checking of the called method itself.
Here, some already existing code is used to verify the function call arguments,
which logs errors whenever the arguments and the parameters do not match. 

In the matcher for field accesses, we first verify that the receiver of the operation
is a class and log an error if it is not.
Then, we check if the class really contains such a field in its declaration.

For the creation of new objects we only need to verify that a corresponding
class declaration exists.

We modified the matcher for new array creations to analyze multidimensional arrays.
The new addition is an iterative loop that verifies
that each dimension's length, which is encoded as an expression, is an integer.