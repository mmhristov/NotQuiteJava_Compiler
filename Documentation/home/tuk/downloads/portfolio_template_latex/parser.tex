\section{Parser}
\label{sec:parser}
In order for the compiler to be able to accept interfaces and 
multidimensional arrays it was for the most part only necessary to extend the grammar in the \textit{notquitejava.cup} file,
adhering to the given grammar in the task description document.
This section is divided into two subsections: Interfaces and Multidimensional Arrays,
each of which covers the important modifications made to the parser.

\subsection{Interfaces}
For the interface extension it was first necessary to add a new
non-terminal symbol to the grammar, which represents the interface declaration,
which can be observed in the following code snippet.
\lstinputlisting[linerange={192-207}, label={code:interfaceDecl}]{"../../../../../src/main/java/frontend/notquitejava.cup"}
From the code snippet we can see that an interface declaration contains interface members, which
are nothing more than function declarations without a body (or implementation).

Next, in order to make interface declarations available in our parser it was needed
to add the non-terminal \textit{interface declarations} to the other global declarations of our program (e.\@.g.\@. classes and functions).
\lstinputlisting[linerange={167-173}, label={code:globalDecls}]{"../../../../../src/main/java/frontend/notquitejava.cup"}

What is left is to modify class declarations so that classes can implement interfaces.
This is shown in the following lines of code:
\lstinputlisting[linerange={175-190}, label={code:classDecl}]{"../../../../../src/main/java/frontend/notquitejava.cup"}

\subsection{Multidimensional Arrays}
For making multidimensional arrays available in our parser, we needed to modify the
existing non-terminals to accept multiple sizes for each dimension.
\lstinputlisting[linerange={344-357}, label={code:arraysCup}]{"../../../../../src/main/java/frontend/notquitejava.cup"}

Other than the modifications in the \textit{.cup} file, we made some necessary adjustments
in \textit{AstHelper.java}.
\lstinputlisting[language=java, linerange={123-132}, label={code:arraysHelper}]{"../../../../../src/main/java/frontend/AstHelper.java"}
Here the multidimensional arrays are created using the generated \textit{NQJ}
classes, iterating through the given dimensions. In order to express each dimension's
size, we added a new field in the \textit{NewArray} object, by modifying the \textit{notquitejava.ast} file.
As the translation of multidimensional arrays was not finished at the time of writing,
it is possible that the order of the created arrays' dimensions is inverted.
This, however, does not put the analysis phase in jeopardy as it is consistent.