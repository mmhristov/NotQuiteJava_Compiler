\section{Translation}
\label{sec:translation}
As the translation of multidimensional arrays was not finished
during the examination period and at the point of writing,
only the translation of classes and interfaces will be presented here.

\subsection{Classes}
\subsubsection{Class Translation and Default Constructor}
\label{subsubsec:classTranslation}
The entry point for the translation of classes is the function \textit{translate()}
in the file Translator.java.
We translate all classes before everything else (e.g.\@ global functions, main function), so that
classes are known and ready to be used.  
The translation of classes consits of three main parts, which can be observed in the following code snippet:
\lstinputlisting[language=java, linerange={86-102}, label={code:translateClasses}]{"../../../../../src/main/java/translation/Translator.java"} 

The method \textit{initClass()} is responsible for the initialization and storing of
its corresponding \textit{TypeStruct} object.
This is done before the actual translation, since we want to make
all classes known beforehand in order to avoid the problem of translating 
a class, which has an unknown class as a field.
Next, for the same reason we initialize all structures, required for
the methods of a class, so that similar
errors are avoided.
After that, the \textit{translateClass()} method is called, where 
the main translation of a class happens.
In this function, first the V-Table is initialized and filled, along with all
its corresponding internal datastructures.
Then, we retrieve the already initialized. but empty \textit{TypeStruct} object
and fill its fields, the first one being the pointer to the V-Table.
This can be obseved in the code snippet below:
\lstinputlisting[language=java, linerange={167-182}, label={code:fillFields}]{"../../../../../src/main/java/translation/Translator.java"}
After we have filled the class' \textit{TypeStruct}, we translate all
class methods with the already existing code that translates global
functions, with a minor modification to skip the translation of
the V-Table pointer.
Lastly, the default constructor for the class is generated, which allocates
memory for the class and provides default values for its fields.

\subsubsection{Field Access in Translated Classes}
The field access matcher function can be found in ExprLValue.java.
Its corresponding method retrieves the 
class' \textit{TypeStruct} object that was created in \Cref{subsubsec:classTranslation}
and searches for the field with the given name.
This can be seen in the code snippet below.
\lstinputlisting[language=java, linerange={74-85}, label={code:getField}]{"../../../../../src/main/java/translation/ExprLValue.java"}
Since the original idea was to also handle classes with inheritance, we start the search
for the field from the end of the field list in order to handle field shadowing.
Nonetheless, as we decided to not implement inheritance because of time constraints,
the order of iteration does not matter here.

It is guaranteed by our analysis phase that the field will be present
in the class and so no additional error handling is necessary.
Therefore, a reference to a temporary variable is returned, which contains the address to the
field that is to be accessed.

In addition to the change that was made in the corresponding matcher for field accesses,
we modified the matcher for the variable usage as well.
This enables the user to use a field of a class in any of its methods, without having
to always use the keyword "this" as receiver.

\subsubsection{Method Calls of Translated Classes}
Method calls are handled in the matcher function \textit{case\_MethodCall(NQJMethodCall \textbf{e})} of the file ExprRValue.java.
Let us examine its workflow, which can be separated into three parts.
The \red{first part} finds the pointer to the method that is to be called by
loading it from the V-Table of the class.
The \blue{second part} constructs the required parameters for the method call,
including the "this" object as the first parameter.
The \pink{third part} of the matcher function is the action of calling the method,
found in \red{part one}, with the parameters, constructed in \blue{part two}.

\subsection{Interfaces}
The analysis phase should guarantee that each interface object in the program
is used correctly. Therefore, in the translation phase
it is only a matter of finding and, if required, translating the class,
with which the interface variable is initialized.
For this, it was necessary to pass some additional
information from the analysis phase to the translation phase (file Analysis.java).
What we decided on doing is for each assignment
to an interface variable to save a reference to the class it is assigned to
in the \textit{InterfaceType} object.
This can be observed below:
\lstinputlisting[language=java, linerange={129-143}, label={code:assignmentInterface}]{"../../../../../src/main/java/analysis/Analysis.java"} 

The reference can then be used in the translator to fetch the class
that is assigned to the interface object.
\lstinputlisting[language=java, linerange={488-491}, label={code:translatorInterface}]{"../../../../../src/main/java/translation/Translator.java"}